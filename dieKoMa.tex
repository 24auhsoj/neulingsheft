\chapter{Über die KoMa}

\section{Warum gibt es die KoMa?}

Während des Studiums lernen viele Studierende nur die eigene Hochschule kennen.
Auch die Mitglieder der studentischen Selbstverwaltung machen da meist keine Ausnahme.
Selten ergibt sich die Gelegenheit, über die lokale Situation hinauszublicken und zu sehen, was an anderen Hochschulen besser oder schlechter läuft.
Dabei können gerade solche Ausblicke neue Impulse für Verbesserungen an der eigenen Hochschule bewirken.
Zudem hängt die Qualität des Studiums auch von Rahmenbedingungen ab, die auf Bundes- oder Landesebene geschaffen werden und lokal kaum beeinflussbar sind.

Doch auch neben diesen hochschulpolitischen Themen gibt es noch einen
wichtigen weiteren Grund sich mit anderen Mathematikstudierenden zu treffen:
Mathematik lebt vom Austausch und von einer Diskussion über
die Themen, das Lernen von Mathematik und das „Herumspinnen“ über mathematische
Fragestellungen. Wie könnte das besser gehen, als wenn man fünf Tage lang Mathematikstudierende zu einer Konferenz versammelt?

Aus diesem Grund gibt es die Konferenz der deutschsprachigen Mathematikfachschaften, kurz KoMa.


\section{Was macht die KoMa?}
„Konferenz der deutschsprachigen Mathematikfachschaften“ klingt sehr förmlich, bezeichnet aber eine lockere Zusammenkunft.

Wir treffen uns für ein paar Tage, diskutieren über Aspekte des Fachs die sonst so im Studium nicht vorkommen,
über Hochschulpolitik und Fachschaftsarbeit und über alles worüber wir gerne reden möchten.
Dieser Austausch findet mit Fachschaften aus ganz Deutschland, Österreich und der Schweiz statt.

Dadurch, dass das ganze Programm freiwillig ist und die Teilnehmenden nur machen, worauf sie Lust haben,
finden auf jeder Konferenzen viele produktive Arbeitskreise statt.

Solche Arbeitskreise tauschen sich beispielsweise über die Gestaltung von Orientierungswochen oder der Fachschaftszeitschrift aus,
über die formale oder inhaltliche Ausgestaltung von Studiengängen oder über Studiengangsakkreditierungen.
% TODO: andere BSPe?

Die Themen werden zwar schon vorher angekündigt, aber es wird erst vor Ort entschieden, welche Themen diskutiert werden.
Auch völlig neue, spontane Arbeitskreise bilden sich gelegentlich.
Daneben haben wir natürlich auch eine Menge Spaß,
lange Abende in den Kneipen, bei Spielen,
oder bei netten Unterhaltungen mit Gleichgesinnten.