\chapter{Handzeichen}\label{sec:handzeichen}

Handzeichen dienen dazu, einen reibungslosen Ablauf der KoMa zu gewährleisten.
Das funktioniert, indem für jeden üblicherweise störenden Zwischenruf ein Handzeichen definiert wurde, welches lautlos gezeigt werden kann und damit jedem die eigene Position anzeigt.
Beim Erstiplenum werden die wichtigen Handzeichen von der Plenumsleitung gezeigt.

\paragraph{Hinweis} \emph{Aufpassen sollte man aber, wenn man ab und an unterschiedliche Konferenzen besucht. So haben allein KoMa und KIF recht unterschiedliche \enquote{Dialekte} die gerne für Verwirrung sorgen.}

\begin{handzeichen}[.2\linewidth]{Melden}{meldung}
Wenn man mal etwas sagen möchte, einfach melden. Die Redeleitung wird der Reihe nach das Wort erteilen.
\begin{itemize}
	\item „Ich möchte etwas sagen.“
\end{itemize}
\end{handzeichen}

\begin{handzeichen}{Ein Satz dazu}{dazu}
Beide Fäuste werden mit ausgestreckten Zeigefingern auf einander zu nach unten bewegt.
Der Redebeitrag wird außerhalb der Redeliste sofort aufgerufen, besteht nur aus einem Satz, und darf nicht mit weiteren \enquote{Ein Satz dazu} beantwortet werden. 
\begin{itemize}
	\item \enquote{Ich will genau dazu jetzt was sagen.}
	\item \enquote{Das ist belegbar falsch und es hilft der Diskussion, wenn ich das jetzt klarstelle.}
\end{itemize}
\end{handzeichen}

\emph{Sollte nur verwendet werden, wenn der Redebeitrag nur genau zu diesem Zeitpunkt sinnvoll erfolgen kann und die Diskussion wesentlich ergänzt. Bitte geht hiermit sparsam um und missbraucht es nicht für „normale“ Redebeiträge.}

\begin{handzeichen}{Technical}{meta}
Beide Händen formen ein \enquote{T} .
Das kann sich z.\,B. auf den Ablauf einer Diskussion beziehen und wird bevorzugt behandelt.
\begin{itemize}
	\item \enquote{Etwas funktioniert gerade nicht}, z.B.: Protokoll
	\item \enquote{Können wir Regel XY bitte einhalten?}
	\item \enquote{Ich möchte ein Verfahren vorschlagen, wie wir weitermachen}, z.B.:
	\begin{itemize}
	    \item \enquote{Die Diskussion hängt hier, können wir eine kurze Pause machen um in Ruhe nachzudenken.}
    	\item \enquote{Ich schlage vor, die Diskussion zu verschieben.}
	    \item \enquote{Können wir ein Fenster öffnen?}
	\end{itemize}
\end{itemize}
\end{handzeichen}

\begin{handzeichen}{Zustimmung}{zustimmung}
Das Wackeln mit beiden Händen symbolisiert Zustimmung 
\begin{itemize}
	\item „Ich bin der gleichen Meinung.“
	\item „Ich finde das gut.“
	\item „Genau so ist das!“
\end{itemize}
\end{handzeichen}

\begin{handzeichen}{Dagegen}{dagegen}
Eine nach oben gestreckte Faust symbolisiert Ablehnung
\begin{itemize}
	\item „Ich bin Dagegen.“
	\item „Ich habe eine ganz andere Meinung dazu.“
\end{itemize}
\end{handzeichen}

\begin{handzeichen}{Veto!}{veto}
Zwei in der Luft gekreuzte Arme mit zu Fäusten geballten Händen bedeuten „Veto!“.
\begin{itemize}
	\item „Mach so weiter und irgendwann verlass ich den Raum.“
	\item bei Abstimmungen: „Veto!“
\end{itemize}
\end{handzeichen}

\begin{handzeichen}[.2\linewidth]{Schildkröte}{kröte}
Die Hand zu einer Schildkröte formen. Das Handgelenk nach vorne kippen und metaphorisch eine Schildkröte umfassen. Sobald man das Handzeichen zeigt, sollte man leise sein.
Immer dann verwenden, wenn es laut und/oder zu unruhig im Raum ist.
\begin{itemize}
	\item \enquote{Es ist hier zu laut.}
	\item \enquote{Schildkröten können nicht reden. Machs Ihnen nach!}
\end{itemize}
\end{handzeichen}

\emph{
    Dieses Handzeichen funktioniert dadurch, dass es nachgemacht wird.
    Sobald es still ist, kann die Hand wieder heruntergenommen werden.
}

\begin{handzeichen}{Langsamer}{langsamer}
Beide Hände mit nach unten zeigenden Handflächen werden gleichzeitig nach unten bewegt. Gerade bei starken Akzenten gerne verwendet.
\begin{itemize}
	\item \enquote{Bitte nicht so schnell.}
	\item \enquote{Bitte rede deutlicher.}
\end{itemize}
\end{handzeichen}

\begin{handzeichen}{Lauter}{lauter}
Beide Hände werden mit nach oben zeigenden Handflächen gleichzeitig nach oben bewegt.
Wenn der Redner zu leise spricht.
\begin{itemize}
	\item \enquote{Bitte sprich ein wenig lauter.}
	\item \enquote{Bitte rede deutlicher.}
\end{itemize}
\end{handzeichen}

\begin{handzeichen}{Du verwirrst mich}{verwirrt}
Die Hand wird vor das Gesicht gehalten und mit den Fingern gewackelt. Der Redner sollte beim Aufkommen dieses Zeichens versuchen sich klarer auszudrücken.
\begin{itemize}
	\item \enquote{Du verwirrst mich.}
	\item \enquote{Fass das bitte nochmal in andere Worte.}
\end{itemize}
\end{handzeichen}

\begin{handzeichen}{Wiederholung}{wiederholung}
Die Hände umkreisen sich mit ausgestreckten Zeigefingern. 
Diese Argumente sind schon bekannt und sollten nicht wiederholt werden.
\begin{itemize}
	\item \enquote{Exakt das wurde schon gesagt.}
	\item \enquote{Du wiederholst dich.}
	\item \enquote{Bitte mit dem nächsten Argument weitermachen.}
\end{itemize}
\end{handzeichen}
