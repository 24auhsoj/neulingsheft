\chapter{Handzeichen}\label{sec:handzeichen}

Handzeichen dienen dazu, das Plenum leise zu halten. Das funktioniert, indem
für jeden üblicherweise störenden Zwischenruf ein Handzeichen definiert wurde,
welches lautlos gezeigt werden kann und damit jedem die eigene Meinung anzeigt.
Bei großen Plena werden die wichtigen Handzeichen meist von der Redeleitung
wiederholt, damit sie auch von allen gesehen werden.

\paragraph{Hinweis} \emph{Aufpassen sollte man aber, wenn man ab und an
unterschiedliche Konferenzen besucht. So haben allein KoMa und KIF recht
unterschiedliche „Dialekte“ die gerne für Verwirrung sorgen.}

\section{Zustimmung}
Das Wackeln mit beiden Händen symbolisiert %% Wackeln → Wedeln?
besonders große Zustimmung (sprich: man hebt trotz Müdigkeit auch die zweite
Hand ©) %%TODO: smiley
\begin{itemize}
	\item „Ich bin der gleichen Meinung“
	\item „Dem stimme ich zu“
	\item „Genau so ist das!“
\end{itemize}

\section{Veto!/Dagegen}
Dieses Handzeichen wird situationsabhängig für „Dagegen“ oder für „Veto“
verwendet.
\begin{itemize}
	\item „Ich bin Dagegen“
	\item „Ich habe eine ganz andere Meinung dazu“
	\item „Mach so weiter und irgendwann verlass ich den Raum“
	\item bei Abstimmungen: „Veto!“
\end{itemize}
\emph{Bei diesem Zeichen muss man immer auf den Kontext achten! Sonst kommt es
zu bösen Missverständnissen.}

\section{Melden}
Wenn man mal etwas sagen möchte, einfach melden. Die Redeleitung wird der Reihe
nach das Wort erteilen.
\begin{itemize}
	\item „Ich möchte etwas sagen“
\end{itemize}

\section{Meta-Meldung/Technical}
Meta-Meldungen können sich z.\,B. auf den Ablauf einer Diskussion beziehen und
sollten bevorzugt behandelt werden. Dies wird durch das „Technical“ erledigt
(beide Hände formen zusammen ein „T“ wie „Timeout“).
\begin{itemize}
	\item „Ich habe eine Meta-Meldung“
	\item „Ich möchte ein Verfahren vorschlagen, wie wir weitermachen“
\end{itemize}

\begin{handzeichen}{Langsamer}{langsamer.png}
Gerade bei starken Akzenten gerne verwendet.
\begin{itemize}
	\item „Bitte nicht so schnell“
	\item „Bitte rede ein wenig langsamer“
\end{itemize}
\end{handzeichen}

\begin{handzeichen}{Lauter}{lauter.png}
Wenn der Redner zu leise spricht.
\begin{itemize}
	\item „Bitte sprich ein wenig lauter“
\end{itemize}
\end{handzeichen}

\section{Du verwirrst mich}
Einfach mit der Hand vor dem Gesicht herumgewedeln (wie wenn man ausdrücken
wollte: „Du spinnst wohl“). Der Redner sollte beim Aufkommen dieses Zeichens
versuchen sich klarer auszudrücken.
\begin{itemize}
	\item „Du verwirrst mich“
\end{itemize}

\begin{handzeichen}{Genau dazu}{dazu.png} %% Genau dazu → Korrektur
„Genau dazu“ wird durch das Bewegen beider offenen Hände zu einem V nach vorne
ausgedrückt.
\begin{itemize}
	\item „Ich will genau dazu jetzt was sagen“
	\item „Dringlichkeitsmeldung“
\end{itemize}
\end{handzeichen}
\emph{Sollte nur verwendet werden, wenn der Redebeitrag nur genau zu diesem
Zeitpunkt sinnvoll erfolgen kann. Bitte geht hiermit sparsam um und missbraucht
es nicht für „normale“ Redebeiträge.}

\section{Fuchs} %%TODO: Schildkröte
Die Hand einfach zu einem Fuchs formen. Immer dann, wenn es laut und zu unruhig
im Raum wird.
\begin{itemize}
	\item „Es ist hier zu laut“
	\item „Der Fuchs ist still und spitzt die Ohren“
\end{itemize}
\emph{Dieses Handzeichen funktioniert durch den Schneeballeffekt. Wenn man
einen Fuchs sieht macht man das Zeichen nach und ist natürlich auch leise. In
sekundenschnelle wird es dann still und man kann die Hand wieder herunter
nehmen. Das Handzeichen sollte nicht als Angriff auf „Störenfriede“ verwendet
werden, sondern signalisieren, dass man gerne mehr mitbekommen möchte.}

\section{Wiederholung}
Diese Argumente sind schon bekannt und sollten nicht wiederholt werden.
\begin{itemize}
	\item „Exakt das wurde schon gesagt“
	\item „Du wiederholtst Dich“
	\item „Bitte mit dem nächsten Argument weitermachen“
\end{itemize}
