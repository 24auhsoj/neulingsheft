\chapter{Einleitung}

Herzlich Willkommen zu Eurer ersten KoMa! Wir hoffen, Ihr seid gut angekommen,
hattet schon den ein oder anderen Kaffee oder Tee und habt sicher auch schon
die ersten anderen Fachschaftler kennen gelernt. Damit das ganze drum-herum mit
\emph{Ewigen Frühstück}, \emph{Arbeitskreisen} und \emph{Plena} ein wenig
klarer für Euch wird, gibt es dieses Heft ganz speziell für Neulinge auf der
KoMa. In diesem Heft möchten wir Euch erklären, was die KoMa „ist“, welche
Begriffe sich über die Jahre eingebürgert haben, warum Handzeichen in den Plena
verwendet werden und noch vieles mehr. Da sich öfters aber auch mal Handzeichen
oder Traditionen ändern, ist das Heft natürlich auch für alle Alt-KoMatiker da.

Neben dem Neulingsheft erhaltet Ihr übrigens noch das Infoheft. Das Infoheft
bekommen alle Konferenzteilnehmer und es erläutert den Konferenzablauf und
erklärt viele organisatorische Dinge. Mit beiden Heften zusammen solltet ihr
gut versorgt sein, sodass Ihr locker und fröhlich loslegen könnt und eine
produktive Zeit in den Arbeitskreisen verbringt, mit den anderen
Konferenzteilnehmern diskutiert und mit viel Energie und Ideen geladen wieder
von dieser Konferenz abreist.

Dieses Neulingsheft ist eine grundlegende Überarbeitung und
Neu"=Zusammenstellung der KoMa"=Materialien der letzten 15~Jahre. %TODO: Update automagically
Da dort zahlreiche Menschen mitgewirkt haben, ist es schwer alle zu nennen.
Großen Dank aber an Nico, Gesa und Fritz (auch wenn ihr schon lange nicht mehr
dabei seid).

\bigskip

\emph{Andreas [CoLa], Universität Paderborn} \\
\emph{im Frühjahr 2012}
