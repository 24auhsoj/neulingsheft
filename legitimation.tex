\chapter{Legitimation der KoMa für politische Themen}

\emph{Ausschnitt aus dem Protokoll einer Diskussion auf der KoMa in Freiburg
(SS~2000)}

Die Frage wird in den Raum gestellt, ob die KoMa politische Themen behandeln
kann, die nicht hochschulspezifische sind?

Folgende Bedenken werden geäußert:
\begin{enumerate}
\item Nicht alle Teilnehmer der KoMa sind von einem gewählten studentischen
	Gremium an die KoMa entsandt worden, d.\,h.\ nicht alle haben eine
	Legitimation, für die Studierendenschaft ihrer Hochschule zu sprechen.  Es
	ist nicht erkennbar, dass alle Teilnehmenden die Meinung der Studierenden
	ihrer Hochschule vertreten.
\item Die Fachschaftsvertretenden sind in den meisten Bundesländern nur mit
	hochschulpolitischem Mandat ausgestattet.  Insbesondere haben einige der
	Vertretenden die auf der KoMa anwesend sind, mit ihren jeweiligen
	Fachschaften politische Positionen nur bezüglich Themen abgesprochen, die
	direkt mit Hochschulpolitik zusammenhängen.
\item Nur ein kleiner Teil der Universitäten ist auf der KoMa im SS~2000
	vertreten. Fachhochschulen fehlen ganz.
\end{enumerate}

Folgende Antworten werden in der Diskussion gegegben:

\begin{enumerate}
\renewcommand\labelenumi{zu \theenumi:}
\item In den meisten Bundesländeren steht der Begriff „Fachschaft“ für alle
	Studierenden des Fachbereichs\footnote{Der Begriff wird in diesem
	Protokoll synonym verwendet für „Fakultät“, „Institut“ und alle anderen
	Bezeichnungen für die Organisationseinheiten, die die Mathematik in den
	verschiedenen Universitäten bilden.} Eine Ausnahme bildet z.\,B.
	Sachsen-Anhalt, wo nicht alle Studierenden automatisch Mitglied der
	Fachschaft sind. Nach dem 2.~Semester dort \emph{können} die
	Studierenden gegen einen Jahresbeitrag von 11\,DM in die Fachschaft
	eintreten. %%TODO: Lustiger Fun Fact, dass das mal so war, ist aber
	vielleicht nicht mehr so relevant …

	Umgangssprachlich wird der Begriff „Fachschaft“ oft für die „aktive
	Fachschaft“ verwendet, also für die Studierenden, die sich für
	studentische Zwecke engagieren. In noch engerem Sinne ist manchmal der
	Fachschaftsrat gemeint.\footnote{Der Begriff wird in diesem Protokoll
	für das gewählte studentische Gremium auf Fachbereichs-Ebene synonym
	verwendet, auch wenn dieses in vielen Universitäten anders heißt.}

	Auf den KoMata der vergangenen Zeit hat sich dazu ein Konsens
	herausgebildet, der die erste Bedeutung bevorzugt. Die KoMa ist also
	eine Konferenz für insbesondere (aber nicht ausschließlich) \emph{alle}
	derzeitigen und ehemaligen Mathematik-Studierenden. Teilnehmende
	können, soweit sie nicht auf eigenen Antrieb an der KoMa teilnehmen,
	von ihren Fachschaftsräten oder anderen Gremien an den Universitäten
	gebeten oder delegiert werden, um an der KoMa teilzunehmen. Sie müssen
	dort aber weder die Meinung der Studierenden ihrer Hochschule vertreten
	noch die des Gremiums, das sie geschickt hat. Vielmehr sind alle
	Teilnehmenden als Privatpersonen auf einer KoMa. Entscheidungen einer
	KoMa sind allein Entscheidungen der Teilnehmenden. Die Teilnehmenden
	sind keine Delegierten ihrer Fachschaften sondern lediglich
	Verbindungsleute.

	Aus diesem Grund werden Entscheidungen auch nie von \emph{der KoMa}
	getroffen, sonderen stets von \emph{der} KoMa im SS~2000 usw. Es sind stets
	die Entscheidungen von genau dieser einen Konferenz. In diesem Sinne
	benötigen die Teilnehmenden der KoMa gar kein Mandat von
	irgendjemandem. Allerdings sollte bedacht werden, dass die Fahrt- und
	Tagungskosten teilweise von den ASten oder den Fachschaften
	rückerstattet werden. Dies bedingt evtl.  eine gewisse Verpflichtung
	zumindest als Mittler zwischen Fachschaft und KoMa zu fungieren.

\item Da die Teilnehmenden einer KoMa lediglich als Privatpersonen an der
	Konferenz teilnehmen, kann jede KoMa zu beliebigen Themen Diskussionen
	führen und Entscheidungen treffen. Darüber hinaus ist es nicht ganz
	klar, ob es tatsächlich eine klare Grenze zwischen Hochschul-Politik
	und Nicht-Hochschul-Politik gibt. Daher muss bei jedem Thema, das auf
	einer KoMa angesprochen wird, immer wieder im Einzelfall eintschieden
	werden, ob sich die KoMa mit diesem Thema befassen möchte oder nicht.

\item Die Einladung wurde an alle Fachschaften an Universitäten in
	Deutschland, Österreich und der Schweiz verschickt, von denen eine
	Adrese bekannt ist, und zwar sowohl per E-Mail als auch in Papierform.
	Daher haben alle Mathematik-Studierenden aller Universitäten die
	Möglichkeit an der KoMa teilzunehmen -- allerdings nur, wenn der
	Informationsfluss innerhalb der Mathematik-Fachbereiche funktioniert.
	Dies kann aber nur in der Verantwortung der studentischen Gremien vor
	Ort liegen.  [\dots]
\end{enumerate}
