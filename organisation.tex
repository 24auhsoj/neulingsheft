\chapter{Interne Organisation}
Damit Teilnehmende sich während einer KoMa treffen, diskutieren und austauschen können, muss diese natürlich auch organisiert werden.
Deshalb werden bei der jeweiligen KoMa die ausrichtenden Fachschaften für die kommenden KoMata bestimmt.
Um zusätzlich ständige Kontaktmöglichkeiten zu haben, gibt es das \emph{KoMa-Büro} und den \emph{Förderverein}.

\section{Das KoMa-Büro}
Das Büro regelt den Briefverkehr der KoMa und pflegt die Mailinglisten.
Es steht mit dem Studentischen Akkreditierungspool (siehe \autopageref{itm:pool}) und seinen von uns entsandten Mitgliedern in Kontakt.
Man kann sich an das Büro mit Fragen zur KoMa wenden und auch externe Anfragen werden vom Büro bearbeitet.

Eine weitere wichtige Aufgabe des KoMa-Büros ist neben dem Alltagsgeschäft die Pflege des KoMa-Archives.
Dieses wurde um 2002 herum ausgiebig recherchiert und beinhaltet sämtliche bekannten Dokumente zur Geschichte der KoMa, so eine Sammlung sämtlicher Publikationen, Protokolle, Teilnahmelisten und sonstiger Erzeugnisse der vergangenen KoMata.

\section{Der Förderverein}
Der \emph{Förderverein der KoMa e.\,V.} wurde 2008 in Paderborn zur finanziellen Unterstützung der KoMata gegründet.
Er kann als gemeinnütziger Verein Förderanträge insbesondere beim Bundesministerium für Bildung und Forschung (\enquote{BMBF}) stellen und Spendenquittungen ausstellen.  % TODO: enquote BMBF
Weitere Informationen sind auf der Webseite \url{https://die-koma.org/foerderverein/} zu finden.

\section{Hinter den Kulissen}
Ein Teil der Organisation, meist die Anmeldung, läuft über \url{https://orga.fachschaften.org}.
Geboten wird dort eine Organisationsplattform die für die KoMata, das Büro und ihre Arbeitskreise verwendet werden kann.
Ein Teil des KoMa-Archiv wird dort verwaltet und auch der Förderverein führt seine interne Organisation über diese Plattform.