\chapter{Interne Organisation}

Wenn man von weitem auf die KoMa guckt, dann könnte man
eigentlich denken: „Die treffen sich, diskutieren, fällen Beschlüsse,
gehen auseinander und merkwürdigerweise kommen die alle nach
'nem halben Jahr wieder zusammen.“ – Und eigentlich stimmt
das sogar im Großen und Ganzen. Man könnte sagen, dass sich
die KoMa durch die KoMa organisiert.\footnote{Hier sind wir bei dem Henne-Ei-Problem angekommen.} Denn auf einer KoMa
wird die ausrichtende Fachschaft für das nächste Mal bestimmt
und alle wissen, dass er seine Arbeit machen wird und auch
wirklich zur nächsten KoMa einlädt. Seit über 30~Jahren hat %%TODO: Update (automagically)
das auch ziemlich gut funktioniert.
Doch dieses ist für „Externe“ ein nur schwer zugängliches System.
Daher und auch um der KoMa aber ein wenig mehr Beständigkeit
zu geben, gibt es zwei wichtige Einrichtungen:

\begin{enumerate}
	\item Zum einen ist dieses das \emph{KoMa-Büro}, welches sich um den
		gesamten Briefverkehr kümmert und der offizielle Ansprechpartner nach
		außen ist.
	\item Zum anderen gibt es seit 2008 den \emph{Förderverein der KoMa
		e.\,V.}, welcher sich um sämtliches Finanzielles kümmert.
\end{enumerate}

\section{Das KoMa-Büro}
Grob zusammengefasst ist es die Aufgabe des KoMa-Büro, den Überblick zu
behalten. Im Büro gehen sämtliche Briefe an die KoMa ein, es steht mit dem
Studentischen Akkreditierungspool (siehe \autopageref{itm:pool}) und unseren
eigenen Leuten im Pool in Kontakt, dort werden die Mailinglisten für die
aktiven KoMa-Teilnehmer und alle Mathematikfachschaften gepflegt. Und dorthin
kann man sich auch wenden, wenn man irgendwelche Fragen zur KoMa hat. Das
können z.\,B. Orga-Fragen sein oder auch Anfragen Externer (von der Presse),
die gerne Kontakt mit der KoMa aufnehmen würden. In ständigem Kontakt steht das
Büro natürlich mit dem Homepage-Team und den jeweiligen Orgas, um die Zeit bis
zur nächsten KoMa möglichst reibungslos zu überbrücken.

Eine weitere ganz wichtige Aufgabe des KoMa-Büros ist neben dem Alltagsgeschäft
die Pflege des KoMa-Archives. Das Archiv wurde um 2002 herum ausgiebig
recherchiert und beinhaltet sämtliche bekannten Dokumente zur Geschichte der
KoMa. Das ist etwa eine Sammlung sämtlicher Publikationen, Protokolle,
Teilnahmelisten etc. der vergangenen KoMata, aber auch allerlei Kurioses wie
selbst erstellte Kartenspiele und Brettspiele. Es ist halt alles, was man als
gute Archivare für die Nachwelt aufbewahren möchte. Als KoMatiker kann man
davon übrigens auch eine Kopie (natürlich nur von Datensätzen ohne persönliche
Daten) erhalten.

\section{Der Förderverein}
Der Förderverein ist noch recht jung und wurde erst~2008 in %%TODO: Nicht mehr jung
Paderborn gegründet. Er ist seitdem Ansprechpartner für die finanzielle
Ausstattung der einzelnen KoMata. Er kann als gemeinnütziger Verein
Förderanträge beim Bund stellen und Spendenquittungen austellen. Er leistet
aber auch direkte Unterstützung zur Organisation der KoMa. Auf jeder zweiten
KoMa findet seine Jahreshauptversammlung statt und ihr seid herzlich eingeladen
dort einzutreten und vielleicht über euer Studium hinaus die KoMa zu
unterstützen.

\section{Hinter den Kulissen}
Ein wesentlicher Teil der Organisation läuft über ein Projekt namens
\url{https://orga.fachschaften.org}. Diese Idee wurde zwar auf der KIF von den
Informatikern ins Leben gerufen, aber wir waren seit Anfang an mit dabei. Die
Idee ist es, eine (abgeschlossene) Organisationsplattform für die KoMata, das
Büro und ihre Arbeitskreise zu bieten. Arbeitskreise können dort ganz
unbürokratisch Wikis und Foren erhalten. Das KoMa-Archiv wird dort verwaltet
und auch der Förderverein führt seine interne Organisation über diese
Plattform.
