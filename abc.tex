\chapter{Das ABC der KoMa}

Über die Jahre hinweg haben sich einige Begriffe eingebürgert, die man auf
jeder KoMa wiederfinden kann. Damit Du auch verstehen kannst, wenn die Orgas
von der „Kasse des Vertrauens neben dem Ewigen Frühstück“ sprechen, gibt es
hier eine kleine Einführung in das KoMa"=Vokabular:

\begin{description}
\item[Abschlussplenum] Am letzten Abend der KoMa findet das Abschlussplenum
	statt. Dort stellen die AKs ihre Ergebnisse vor. Beschlüsse, z.\,B. zu
	Resolutionen werden hier gefasst.  Freiwillige für die Ausrichtung der
	nächsten Konferenzen werden spätestens hier rekrutiert/bestimmt. Ein
	solches Abschlussplenum kann auch schon mal 4 Stunden dauern.  Da die
	Dauer aber etwas mit Disziplin, Konsensfähigkeit und Konzentration zu
	tun hat, gibt es auf der KoMa seit der 58.~KoMa in Oldenburg ein
	Alkoholverbot im Plenum.

\item[Adressliste] In der Regel werden auf der KoMa zwei Adresslisten erstellt
	bzw.\ gepflegt. Eine enthält die Adressen der Teilnehmer, meist inklusive
	E-Mail, sortiert nach Vorname oder Ort. Sie wird bei der Anmeldung
	erstellt, bei der Anreise überprüft, an die Teilnehmer verteilt, ist
	aber nicht öffentlich. Als zweites gibt es (meist) eine Liste der
	Fachschaftsadressen. Diese dient vor allem der Korrektur der
	vorhandenen Listen. Die Teilnehmerliste wird in letzter Zeit
	üblicherweise im Rahmen des AK~Networking erstellt.

%\item[AK] AK heißt „Arbeitskreis“. Die meiste inhaltliche Arbeit auf der KoMa
	finden in den Arbeitskreisen statt. Diese werden nicht von der
	ausrichtenden Fachschaft organisiert, sondern von den Teilnehmern
	vorgeschlagen und zum Teil auch vorbereitet. Es besteht der Wunsch,
	Arbeitskreise vor der Konferenz über die \emph{KoMapedia} anzukündigen,
	was in letzter Zeit immer besser klappt. Spontane Vorschläge im
	Anfangsplenum sind aber immer möglich. In den Arbeitskreisen wird das
	Thema je nach Interesse und vorhandenem Material bearbeitet. Die
	Gruppen bestehen meist aus~5 bis 20~Personen. Die Arbeitsweisen gehen
	von Diskussionen über Literaturarbeit bis zu Basteln und künstlerischen
	Aufführungen. Die AKs laufen während der gesamten Konferenz, je nach
	Arbeitskreis auch 24-stündig.

	Die AKs präsentieren sich auf dem Zwischen- bzw.\ Abschlussplenum und möglichst
	auch mit einem Artikel im KoMa-Kurier. Eine Ansprechperson und mögliche
	Orte/ Termine für Zwischentreffen (siehe WAchKoMa) werden ebenfalls auf
	dem Abschlussplenum bekanntgegeben.

	Typische AK-Themen die häufiger vorkamen/-kommen sind beispielsweise:
	\begin{itemize}
		\item Bachelor/Master
		\item Studiengebühren/-beiträge
		\item Prüfungsordnung
		\item Lehramt
		\item Studentische Veranstaltungsevaluation
		\item Nachwuchswerbung
		\item Fachschaftszeitungen, -comics, -homepages
		\item Jonglieren
		\item AK Pella -- Dichtung und Gesang
	\end{itemize}
	Wie man hier sieht, gibt es also neben den inhaltlichen auch immer
	Freizeit-AKs. Bei den letzten Konferenzen wurde aber auch daran
	erinnert, dass die „Spaß-AKs“ nicht der vordergründige Grund seien
	sollten, warum ihr hier seid.

\item[Anfangsplenum] Mit dem Anfangsplenum beginnt offiziell die Konferenz. Es
	wird meist auf Mittwoch 20~Uhr angesetzt.  Häufig beginnt es allerdings
	etwas später, da viele Personen oft später anreisen als erwartet.

	Dort gibt die ausrichtende Fachschaft zunächst organisatorische Hinweise. Dann
	wird von jeder vertretenen Hochschule kurz berichtet, was in der
	jeweiligen Fachschaft und an der jeweiligen Hochschule gerade läuft und
	wer von dort auf der Konferenz ist. Zudem gibt es Berichte über die
	hochschulpolitische Lage in den verschiedenen Bundesländern. Diese
	werden allerdings auch manchmal in das Zwischenplenum ausgelagert. Dann
	werden Vorschläge für Arbeitskreise (AKs) gesammelt und abgefragt, wie
	viel Interesse jeweils daran besteht. Letztlich wird festgestellt,
	welche AKs überhaupt stattfinden (d.\,h.\ genügend Interesse gefunden
	haben). Diesen werden dann Zeiten und Räume zugeteilt. Eine
	verbindliche Anmeldung zu den AKs erfolgt nicht.

\item[Anmeldung] In der Einladung werden die Teilnehmer aufgefordert, sich bei
	der ausrichtenden Fachschaft anzumelden, am einfachsten auf der
	Konferenzwebseite \url{https://die-koma.org}. Die Anmeldung sollte zeitig
	genug erfolgen, so dass die ausrichtende Fachschaft noch genügend T-Shirts
	bestellen und das Essen besser planen kann.

	Wer auf der Konferenz eintrifft, meldet sich beim Orgabüro.  Diese Meldung
	besteht normalerweise aus: freudiger Begrüßung, Teilnahmebeitrag bezahlen,
	Quittung/""Teilnahmebestätigung erhalten, Adressenliste überprüfen, Namensschild
	herstellen oder anstecken, evtl.\ Tagungsticket erhalten/kaufen, evtl.\
	Programmheft/""Kulturheft/""Stadtplan mitnehmen.

\item[Ausrichtende Fachschaft] Eine Fachschaft übernimmt die Planung und
	Organisation der Konferenz. Dazu gehört jedoch nicht die inhaltliche
	Vorbereitung. Soweit wie möglich wird auf jeder Konferenz schon die
	ausrichtende Fachschaft für die übernächste Konferenz (quasi „in einem
	Jahr“) bestimmt.

\item[Beschlüsse] Beschlüsse der KoMa werden vom Plenum gefasst und sind
	Beschlüsse der anwesenden Personen. Sie erheben weder den Anspruch, alle
	Fachschaften (oder alle auf der Konferenz vertretenen Fachschaften) zu
	repräsentieren, noch für alle folgenden Konferenzen verbindlich zu sein.
	Letzteres ergibt sich daraus, dass die nächste Konferenz sich ja aus
	anderen Personen zusammensetzt. Trotzdem gibt es Beschlüsse, die die
	Organisation der Konferenzen betreffen und die zumindest als dringende
	Empfehlung an die ausrichtende Fachschaft zu verstehen sind. Schließlich
	sind viele, die den Beschluss mitgetragen haben, beim nächsten Mal wieder
	dabei. Beschlüsse werden nach dem Konsensprinzip gefasst (siehe „Konsens“).

\item[Einladung] Längere Zeit vor den Konferenzen verschickt die ausrichtende
	Fachschaft Einladungen über Mailinglisten und per Post an alle bekannten
	Fachschaften, soweit deren Adressen bekannt sind. Darin wird vor allem der
	Termin bekanntgegeben, aufgefordert sich anzumelden und AKs vorzuschlagen.
	Etwas dichter vor den Konferenzen gibt’s dann noch eine Erinnerung via
	E-Mail. Hier sind vor allem die Wegbeschreibung und der genaue
	Anfangszeitpunkt enthalten, ein Hinweis auf die Höhe des Teilnahmebeitrags
	sowie weitere organisatorische Details. Da sich (leider) nicht alle
	Teilnehmer (rechtzeitig) anmelden, ist es wichtig, dass insbesondere die
	Wegbeschreibung über Internet zugänglich ist.

\item[Essen/„Ewiges Frühstück“] Von der Orga-Mann-und"=Damenschaft
	bereitgestellt und im Teilnehmerbeitrag enthalten ist das „Ewige
	Frühstück“. Dieses besteht aus einem Buffet mit Brot/Semmeln,
	Margarine/Butter, Marmelade, Käse, Müsli, Milch, Obst, Gemüse, etc. Dort
	bedienen sich alle selbst. Es steht den ganzen Tag über zur Verfügung.
	Darüber hinaus gibt es meist eine warme Mahlzeit am Tag -- ein vegetarisches
	Essen ist immer dabei. Freitags geht man meistens zusammen in der Mensa
	essen. Auf Sommerkonferenzen wird oft gegrillt.

	Oft stellt das Orga-Team neben dem „Ewigen Frühstück“ auch Müsli- und
	Schokoriegel zur Verfügung. Diese werden über eine Strichliste abgerechnet
	und am Ende der Konferenz in der „Kasse des Vertrauens“ bezahlt.

\item[Förderverein der KoMa e.\,V.] Der „Förderverein der KoMa~e.\,V.“ wurde
	auf der 63. KoMa in Paderborn gegründet. Er ist gemeinnützig und sein Ziel
	ist es, die Organisation der KoMa zu unterstützen. Als solcher kümmert er
	sich um Spenden, Anträge auf Bundesfördermittel oder auch um Sponsoren.
	Jeder, der die KoMa unterstützen will, kann gerne dem Verein beitreten. Die
	Vereinssitzungen finden üblicherweise während der KoMa statt.

\item[Geschäftsordnung] Die KoMa hat keine Geschäftsordnung oder Satzung -- aus
	Prinzip. Bürokratische und inhaltsleere formale Strukturen werden
	abgelehnt. Verfahrensweise und Struktur können sich auf jeder Konferenz
	daher ändern.

\item[Getränke] Kaffee, Tee, Milch und Wasser gehören zum Frühstück und müssen
	nicht extra bezahlt werden. Weiter gibt es Bier, Saft und gelegentlich
	Wein. Diese werden, wie die Schokoriegel, über eine Strichliste abgerechnet
	und am Ende der Konferenz in der „Kasse des Vertrauens“ bezahlt.

\item[Handzeichen] Zur Verbesserung des Diskussionsablaufs wurden Handzeichen
	vereinbart, die z.\,B. Zustimmung oder Ablehnung signalisieren, ohne Krach
	zu machen. Details dazu gibt es auf \autopageref{sec:handzeichen}.

\item[Infoheft] Ein Heft mit langer Tradition: Bei der Anmeldung erhalten die
	Teilnehmer das Infoheft. Darin sind organisatorische Hinweise aufgeführt,
	das Programm, Wegweiser und Tipps für das Abendprogramm. Sofern es den
	nicht separat gibt, gehören auch ein Stadtplan, Kulturübersicht,
	Geschichtsabriss etc.\ zum Infoheft.

\item[Isomathe] Aufrollbare Schlafunterlage für Naturwissenschaftler,
	Sozialpädagen sowie speziell Mathematiker. Wird von den Teilnehmern der
	Konferenzen selbst mitgebracht.

\item[KIF] KIF, [die], Konferenz der (deutschsprachigen)
	Informatikfachschaften.

	Diese bezeichnet in erster Linie die Zusammenkunft der Teilnehmer einmal
	pro Semester. Die Konferenz besteht aus den teilnehmenden Fachschaftler,
	welche aus dem gesamten deutschsprachigen Raum anreisen. Die Nummerierung
	der Konferenzen besteht aus den Jahreszahlen seit der ersten KIF.\@
	Teilnehmer der KIF werden nicht als „Kiffer“, sondern als „KIFfel(s)“
	bezeichnet. KIF und KoMa haben eine lange freundschaftiche Verbundenheit
	und finden öfters sogar zeitglich am gleichen Ort statt.

\item[KoMa] Koma, 1 [die], um den Kern eines Kometen liegende Nebelhülle
	(Gasatmosphäre). -- 2 [die], Bildfehler bei Linsen oder Linsensystemen:
	Seitlich der optischen Achse gelegene Punkte werden nicht punktförmig
	sondern in Form eines Kometenschweifes abgebildet. -- 3 [das], Coma, tiefe
	Bewusstlosigkeit, z.\,B. bei Zuckerkrankheit, Harnvergiftung, u.\,a.\
	(Quelle: Bertelsmann Universallexikon via „Koma für Neulinge“ [Dank an Gesa
	aus Köln])

	Und die wichtigste Bedeutung: -- 4 [die], Konferenz der deutschsprachigen
	Mathematik"=Fachschaften.

	Letztere bezeichnet in erster Linie die Zusammenkunft der Teilnehmer einmal
	pro Semester. Es gibt eine SommerKoMa und eine WinterKoMa. Über den Zusatz
	„deutschsprachigen“ wurde auf der KoMa in Bonn (WS94/95) mal diskutiert,
	mit dem Ziel, nicht nationalistisch zu sein. Aus demselben Grund und auf
	ausdrücklichen Wunsch der Teilnehmer aus Österreich wurde er dann aber
	beibehalten, weil nur so klar wird, dass die KoMa keine reine
	Bundesfachschaftentagung ist. Schließlich kommen regelmäßig Personen aus
	Österreich und der Schweiz. Die erste KoMa war im WS 1977/78\footnote{Der
	Ort konnte bisher nicht festgestellt werden, ein Protokoll existiert
	jedoch.} noch unter dem Namen \emph{VDS"=Fachtagung Mathematik}. In den
	80ern änderte sich dieser Name in \emph{Bundesfachschaftentagung
	Mathematik}, bevor der heutige Name entstand.  Bis 2005 wurde zur
	Nummerierung die Zählung nach Paulus (n.\,P.) verwendet, dem Rekordbesucher
	der KoMa, welcher alle von ihm besuchten KoMata als Grundlagen einer
	Nummerierung wählte. Nach Recherchen im KoMa-Archiv konnte aber die exakte
	Anzahl der stattgefundenen KoMata bestimmt werden\footnote{Die Zählung nach
	Paulus musste dadurch um~6 nach oben korrigiert werden.}. Die korrigierte
	Anzahl wurde als neue Nummerierung übernommen.  Als Pluralbildungen sind
	KoMen, KoMata, KoMas und, seltener, KoMae in Gebrauch. Auf der KoMa~63
	wurde sich aber auf den Begriff „KoMata“ als Pluralform geeinigt, um dieses
	Chaos zu beenden.

\item[KoMa-Archiv] %%TODO: Wäre schön, wenn das so wäre … Vieles hat sich über
	die letzten Jahrzehnte angesammelt. Alle Daten, Publikationen und Akten der
	KoMa werden daher im KoMa-Archiv aufgehoben und gepflegt.  Jeder
	KoMa"=Teilnehmer und jede Mathematikfachschaft kann eine Kopie des digitalen
	KoMa"=Archives beim Büro erhalten. Das Archiv wird vom KoMa-Büro verwaltet
	und gepflegt.

\item[KoMa-Büro] Eine Fachschaft verwaltet die an die KoMa gerichtete Post und
	verschickt den KoMa-Kurier, sofern dieser nicht mit den Einladungen
	verschickt wird. Das KoMaBüro befindet sich zur Zeit an der TU~Chemnitz. %TODO: Update
	Es dient auch als Geschäftsadresse des Fördervereins der KoMa~e.\,V.

\item[KoMa-Kasse] Früher gab es ein virtuelles Säckchen Geld, welches von Orga
	zu Orga weitergegeben wurde und vor allem für die Vorfinanzierung der KoMa
	diente. Heutzutage wird „das Finanzielle“ aber über den Förderverein
	geregelt, welcher auch eine eigene Kasse besitzt. Da der Verein aber außer
	Spenden keine Einnahmen hat, ist er auch auf die Unterstützung der
	Teilnehmer und teilnehmenden Fachschaften angewiesen.

	Ihr könnt euch etwa folgendes überlegen: Wenn ihr auf eine KoMa fahrt und
	aus einer der Hochschulen kommt die eure Auslagen erstatten, dann macht ihr
	ja eigentelich ein Plus, da ihr daheim sowieso Geld für Essen ausgegeben
	hättet.  Wie wäre es also dieses Geld (quasi den Teilnehmerbeitrag) an den
	Förderverein zu spenden und damit die Ausrichtung der kommenden KoMata und
	die Finanzierung von Zwischentreffen zu unterstützen?!

\item[KoMa-Kurier] Der KoMa-Kurier (früher auch KoMa-Kuhrier geschrieben) ist
	eine Art Zeitung, die an möglichst alle Fachschaften verschickt wird. Er
	besteht vor allem aus Protokollen und AK"=Berichten der jeweils letzten
	KoMa, dem legendären Vorwort und allem, was sonst noch Leute so beisteuern.

\item[KoMapedia] Die KoMapedia ist das Wiki der KoMa und wird zur Dokumentation
	von Arbeitskreisen und KoMata genutzt. Sie ist über die KoMa"=Webseite
	die-koma.org erreichbar.

\item[Konsens] Konsens heißt nicht, dass alle einer Meinung sind.  Konsens
	heißt, eine Entscheidung zu treffen, mit der alle leben können. Dabei gibt
	es unterschiedliche Stufen: Das Einfachste ist: alle sind dafür. Weiter
	kann es sein, dass einige dafür sind und einige mehr oder weniger starke
	Bedenken dagegen haben, aber damit leben können, wenn der Beschluss so
	gefasst wird. Vielleicht werden sie sich nicht aktiv an der Umsetzung
	beteiligen. Kein Konsens liegt vor, wenn eine Person ein Veto einlegt. Das
	Veto bedeutet, dass diese Person mit dem Beschluss nicht leben kann und
	vielleicht die KoMa verlassen oder sich nicht mehr zur KoMa gehörig fühlen
	würde, wenn der Beschluss so umgesetzt wird. In diesem Fall ist kein
	Beschluss gefasst.  Es ist aber z.\,B. möglich, dass diejenigen, die etwa
	eine Resolution befürworten, diese jetzt privat unterschreiben und
	veröffentlichen, aber eben nicht als KoMa.

	Konsens ist übrigens nicht gleich Konsens, da gibt es feine Unterschiede
	zwischen KIF und KoMa. Auf der KIF ist es auch möglich bzw.\ üblich einen
	Konsens zu fassen, indem man eine Abstimmung durchführt.

\item[Meinungsbild] Im Plenum wird manchmal gefragt, „Wer ist dafür/wer ist
	dagegen?“, um festzustellen, ob überhaupt Bedarf oder die Möglichkeit
	besteht, eine bestimmte Entscheidung zu treffen. Dies ist kein Beschluss!
	Das Meinungsbild soll lediglich allen die Möglichkeit geben, zu sehen, wie
	die anderen gerade denken. Da es das Konsensverfahren durcheinander bringen
	kann, weil es wie eine Abstimmung aussieht, wird es auch kritisch gesehen.

\item[Namensschild] Bei der Anmeldung basteln sich (oder bekommen) alle ein
	Namensschild. Darauf steht der Vorname und die Hochschule. Das Namensschild
	wird zwecks besserer Kontaktaufnahme während der ganzen Konferenz getragen.

\item[Orga] Wahnsinnige, die einen Moment lang nicht oder zu wenig nachgedacht
	haben. Diejenigen, die die Konferenzen vorbereitet haben und für die
	Organisation zuständig sind.  Oft durch spezielle Namensschilder oder
	T-Shirts gekennzeichnet.

\item[Plenum] Im Plenum treffen sich alle Teilnehmer, um gemeinsam
	Informationen auszutauschen und zu diskutieren. Vom Plenum werden
	Beschlüsse gefasst. Immer gibt es ein Anfangsund ein Abschlussplenum, nach
	Bedarf auch ein oder mehrere Zwischenplenen. Die Teilnahme am Plenum ist
	natürlich freiwillig, trotzdem ist es wichtig, dass möglichst alle daran
	teilnehmen, um Informationen an alle weitergeben zu können und alle
	Positionen berücksichtigt werden können. Bei themenbezogenen Zwischenplenen
	ist das z.\,T. weniger wichtig.

	Führung von Protokoll und Redeliste wird im Zweifel von der ausrichtenden
	Fachschaft organisiert. Die Moderation übernehmen einzelne Teilnehmer nach
	Lust und Laune.

	\paragraph{Achtung} In den Plenen der KoMa herrscht Alkoholverbot!

\item[Protokoll] Dokumentiert Geschehenes sprachlich neutral, objektiv und
	allumfassend. Etwas, vor dem sich der gemeine Konferenzteilnehmer
	üblicherweise drückt, weswegen es auch schwer wird, nachträglich eine
	umfassende Konferenzdokumentation zu erstellen.

\item[Redeliste] Kann bei Bedarf/Wunsch eingeführt werden. Im Plenum werden
	dann die Wortmeldungen auf der Redeliste notiert. Sollte durch eine gute
	Moderation unnötig werden.

\item[Resolution] Eine gemeinsame Stellungnahme der KoMa (d.\,h.\ der dort
	anwesenden Menschen) zu meist politischen Themen wird häufig auf dem
	Abschlussplenum beschlossen.  Diese wird veröffentlicht (Presse) und an
	jeweilige Ministerien/Regierung usw. verschickt. Es besteht der Wunsch,
	dass Resolutionen vor Beginn des Abschlussplenums aushängen, damit alle sie
	lesen können. Traditionell gibt es fast immer mindestens eine Resolution
	auf der KoMa.

\item[Satzung] Siehe Geschäftsordnung.

\item[Schlafquartiere] Zum Schlafen bringen die Teilnehmer Schlafsack und
	Isomathe mit. Wenn möglich gibt es ein gemeinsames Schlafquartier in
	geeigneten Räumen, z.\,B. Turnhalle oder Jugendzentrum. Wenn es nicht
	anders geht, werden die Teilnehmer einzeln oder in kleinen Gruppen bei
	einheimischen Studies oder WGs untergebracht. Frühstück erhalten sie dann
	zentral. Im Allgemeinen sind die Teilnehmer aber nicht sehr anspruchsvoll.
	Nähe zu Frühstücks-/Tagungsraum und gemeinsame Unterkunft wird jedoch
	bevorzugt.

\item[Stadtführung] Die ausrichtende Fachschaft veranstaltet eine Stadtführung.
	Sie wird in der Regel von einheimischen Studies geleitet. Dabei liegt der
	Schwerpunkt nicht unbedingt auf touristischen Attraktionen, sondern auf
	einem Einblick in den Hochschulort und das zugehörige Studieleben.

\item[Strichliste] Neben den Getränken hängt eine große Liste, in die sich alle
	eintragen und für ihre Getränke Striche machen.  Bezahlt wird vor der
	Abreise in die „Kasse des Vertrauens“, die vom Orga-Team am Samstagabend
	aufgestellt wird.  Wasser ist traditionell kostenlos und wird daher nicht
	in die Strichliste eingetragen.

\item[Studentischer Akkreditierungspool]\label{itm:pool} In Deutschland müssen alle Bachelor-
	und Masterstudiengänge akkreditiert werden. An solch einer Akkreditierung
	sind auch immer Studierende beteiligt, welche vom unabhängigen
	„Studentischen Akkreditierungspool“
	(\url{https://www.studentischer-pool.de}) zugeteilt werden. Die KoMa
	entsendet in ihrer Rolle als Bundesfachschaftentagung jeweils
	Mathematikstudierende in diesen Pool.

\item[Tagungsticket] (oder auch „Konferenzticket“) Je nach Möglichkeit und
	Notwendigkeit (Verkehrsangebot, Lage von Schlaf- und Tagungsräumen, Preis)
	gibt es zu den Konferenzen ein Tagungsticket. Dies muss eventuell
	zusätzlich zum Teilnehmerbeitrag bezahlt werden und berechtigt zur
	Benutzung der öffentlichen Verkehrsmittel der jeweiligen Stadt während der
	Konferenzen.

\item[Teilnehmerbeitrag] Zur Finanzierung der Konferenzen
	(Einladungen"=Verschicken, Essen, Unterkunft, Namensschilder, Büromaterial
	für Organisation, etc.; Getränke (bis auf Wasser, Tee, Kaffee) werden
	getrennt abgerechnet) zahlen die Teilnehmer einen Beitrag. Dieser liegt in
	den letzten Jahren immer zwischen 25\,€ und 30\,€. Das Tagungsticket muss
	eventuell extra bezahlt werden. Um das Geld ggf. vom AStA/StuRa/Konvent
	oder der Hochschule erstattet zu bekommen, gibt es eine Quittung.

\item[Teilnehmer] Menschen, die an den Konferenzen teilnehmen.  Zur Teilnahme
	ist es weder Pflicht, einen mathematischen Studiengang zu studieren, noch
	bei irgendeiner Fachschaft aktiv zu sein. Es ist aber üblich.

\item[Termin] Die Konferenzen gehen in der Regel von Mittwoch Abend bis Sonntag
	Vormittag. Die Sommer"=Konferenzen finden meist über einen freien Donnerstag
	Ende Mai/Anfang Juni statt, die Winterkonferenzen etwa Mitte November
	(früher Buß- und Bettag).

\item[T-Shirts] In den letzten Jahren wurden vom Orga-Team immer T-Shirts für
	die Konferenzen bedruckt. Man gibt bei der Anmeldung an, ob man ein T-Shirt
	haben möchte. Manchmal gibt es noch Restbestände, so dass man auch ohne
	Anmeldung kaufen kann. Das T-Shirt wird üblicherweise getrennt vom
	Teilnehmer"=Beitrag bezahlt. Die Kosten bewegen sich meistens zwischen 7\,€
	und 12\,€.

\item[Veto] Wer bei einer Konsensentscheidung mit einem Beschluss überhaupt
	nicht leben kann, kann ein Veto einlegen. Mit einem Veto ist kein
	Konsensbeschluss möglich, es sei denn die KoMa spaltet sich.

\item[WAchKoMa] Moderner Name für „Zwischentreffen“. Es bedeutet „Weiterführung
	von Arbeitskreisen unter chaotischen Verhältnissen der Konferenz der
	deutschsprachigen Mathematikfachschaften“.

\item[Zwischenplenum] Auf der KoMa gibt es zusätzlich zu Anfangsund
	Abschlussplenum weitere Plenen. Dort gibt es Berichte und/oder Diskussionen
	zu speziellen Themen. Themen für Zwischenplenen waren bspw.
	Studiengebühren, neues Hochschulrahmengesetz, BAföG.\ Manchmal wurden auch
	Berichte über die hochschulpolitische Lage in den verschiedenen
	Bundesländern vom Anfangsplenum auf ein Zwischenplenum ausgelagert.

\item[Zwischentreffen] Einige AKs treffen sich auch zwischen zwei Konferenzen
	noch mal und richten eine so genannte „WAchKoMa“ aus. Das Treffen wird von
	den AK"=Mitgliedern selbst organisiert und ist in der Regel auch offen für
	Personen, die auf der Konferenzen nicht in dem AK waren.  Eine grobe
	Planung für Ort und Termin wird meist schon auf dem Abschlussplenum
	bekanntgegeben, genaueres gibt es üblicherweise über die Mailingliste(n).
\end{description}
